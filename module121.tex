\section{Composing Transfer Functions}
\FloatBarrier
\begin{flushright}
    \textit{Scribed by Arpit Saxena}
\end{flushright}

\begin{figure}[H]
    \centering
    \subfloat[\label{fig:m121:fig1:a}] {{
        \tt
        \begin{tikzpicture}[->,
            node distance = 12mm,
            start chain = going below,
            box/.style = {draw,rounded corners,fill=white,
                on chain,align=center,inner xsep=8mm},
            unbox/.style = {on chain,align=center},
            edge/.style = {midway}]
            \coordinate[on chain] (begin);
            \node[box](b1) {f\textsubscript{1}};
            \node[box](b2) {f\textsubscript{2}};
            \coordinate[on chain] (end);
            \node[draw,fit=(b1) (b2)](summary) {};
            \node[below right] at (summary.south) {f\textsubscript{summary}};
            \path[]
                (begin) edge (b1)
                (b1) edge (b2)
                (b2) edge (end);
        \end{tikzpicture}
    }}
    \subfloat[\label{fig:m121:fig1:b}] {{
        \tt
        \begin{tikzpicture}[->,
            node distance = 12mm,
            start chain = going below,
            box/.style = {draw,rounded corners,fill=white,
                on chain,align=center,inner xsep=8mm},
            unbox/.style = {on chain,align=center},
            edge/.style = {midway}]
            \coordinate[on chain] (begin);
            \node[box, inner ysep=2mm](b1) {};
            \node[box, below left=of b1](b2) {f\textsubscript{1}};
            \node[box, below right=of b1](b3) {f\textsubscript{2}};
            \node[box, below=of b1, below left=of b3, inner ysep=2mm](b4) {};
            \coordinate[on chain] (end) {};
            \path[]
                (begin) edge (b1)
                (b1) edge (b2)
                (b1) edge (b3)
                (b2) edge (b4)
                (b3) edge (b4)
                (b4) edge (end);
            \node[draw, fit=(b1)(b2)(b3)(b4)](summary) {};
            \node[below right] at (summary.south) {f\textsubscript{summary}};
        \end{tikzpicture}
    }}
    \subfloat[\label{fig:m121:fig1:c}] {{
        \tt
        \begin{tikzpicture}[->,
            node distance = 12mm,
            start chain = going below,
            box/.style = {draw,rounded corners,fill=white,
                on chain,align=center,inner xsep=8mm},
            unbox/.style = {on chain,align=center},
            edge/.style = {midway}]
            \coordinate[on chain] (begin);
            \node[box](b1) {f};
            \coordinate[on chain] (end) {};
            \path[]
                (begin) edge (b1)
                (b1) edge (end);
            \draw (b1.230) -| coordinate (l1) ++ (0.6,-0.3) -| coordinate (l2) ([xshift=5mm]b1.east) |-
            coordinate (l3) ([yshift=5mm]b1.50) -- coordinate (l4) (b1.50);
            \node[draw, fit=(b1) (l1) (l2) (l3) (l4)](summary) {};
            \node[below right] at (summary.south) {f\textsubscript{summary}};
        \end{tikzpicture}
    }}
    \caption{}


\end{figure}

\begin{itemize}
    \item The summary functions for larger regions are based on the transfer
    functions of the smaller sub-regions.
    \item The summary functions may involve sophisticated symbolic analysis.
    \item The fallback summary transfer functions can be described using
    composition, meet, and closure operators on transfer functions (for 
    reducible graphs).

    The loops can be captures by the structure that there is a loop body and a
    back edge, the value for it is being captured by the closure. Closure
    means to repeat f one or more times, potentially unbounded.
    \item More sophisticated summary transfer functions can often be described
    using composition, meet, and closure operations on transfer functions.
\end{itemize}

\subsection{Transfer Function Operations}

\begin{figure}[H]
    \centering
    \subfloat[\centering Composition \label{fig:m121:fig2:a}] {{
        \tt
        \begin{tikzpicture}[->,
            node distance = 12mm,
            start chain = going below,
            box/.style = {draw,rounded corners,fill=white,
                on chain,align=center,inner xsep=8mm},
            unbox/.style = {on chain,align=center},
            edge/.style = {midway}]
            \coordinate[on chain] (begin);
            \node[box](b1) {f\textsubscript{1}};
            \node[box](b2) {f\textsubscript{2}};
            \coordinate[on chain] (end);
            \node[draw,fit=(b1) (b2)](summary) {};
            \node[right] at (summary.east) {$f_1 \circ f_2$};
            \path[]
                (begin) edge (b1)
                (b1) edge (b2)
                (b2) edge (end);
        \end{tikzpicture}
    }}
    \subfloat[\centering Meet\label{fig:m121:fig2:b}] {{
        \tt
        \begin{tikzpicture}[->,
            node distance = 12mm,
            start chain = going below,
            box/.style = {draw,rounded corners,fill=white,
                on chain,align=center,inner xsep=8mm},
            unbox/.style = {on chain,align=center},
            edge/.style = {midway}]
            \coordinate[on chain] (begin);
            \coordinate[on chain] (dummy);
            \node[box, below left=of dummy](b2) {f\textsubscript{1}};
            \node[box, below right=of dummy](b3) {f\textsubscript{2}};
            \coordinate[on chain, below=of dummy, below left=of b3] (end) {};
            \path[]
                (begin) edge[-] (dummy)
                (dummy) edge (b2)
                (dummy) edge (b3)
                (b2) edge (end)
                (b3) edge (end);
            \node[draw, fit=(dummy)(b2)(b3)](summary) {};
            \node[right] at (summary.east) {$f_1 \wedge f_2$};
        \end{tikzpicture}
    }}
    \subfloat[\centering Closure\label{fig:m121:fig2:c}] {{
        \tt
        \begin{tikzpicture}[->,
            node distance = 12mm,
            start chain = going below,
            box/.style = {draw,rounded corners,fill=white,
                on chain,align=center,inner xsep=8mm},
            unbox/.style = {on chain,align=center},
            edge/.style = {midway}]
            \coordinate[on chain] (begin);
            \node[box](b1) {f};
            \coordinate[on chain] (end) {};
            \path[]
                (begin) edge (b1)
                (b1) edge (end);
            \draw (b1.230) -| coordinate (l1) ++ (0.6,-0.3) -| coordinate (l2) ([xshift=5mm]b1.east) |-
            coordinate (l3) ([yshift=5mm]b1.50) -- coordinate (l4) (b1.50);
            \node[draw, fit=(b1) (l1) (l2) (l3) (l4)](summary) {};
            \node[right] at (summary.east) {f*};
        \end{tikzpicture}
    }}
    \caption{}


\end{figure}

\subsubsection{Example: Constant Propagation}

\begin{minipage}{0.4\textwidth}
    \tt
    \captionsetup{type=figure}
    \begin{tikzpicture}[->,
        node distance = 12mm,
        start chain = going below,
        box/.style = {draw,rounded corners,fill=white,
            on chain,align=center,inner xsep=8mm},
        unbox/.style = {on chain,align=center},
        edge/.style = {midway}]
        \coordinate[on chain] (begin);
        \node[box](b1) {x = 2};
        \node[box](b2) {y = 3};
        \coordinate[on chain] (end);
        \path[]
            (begin) edge (b1)
            (b1) edge (b2)
            (b2) edge (end);
    \end{tikzpicture}
    \captionof{figure}{Constant Propagation: Composition}
    \label{fig:m121:const_prop_composition}
\end{minipage}
\hfill
\begin{minipage}{0.5\textwidth}
    \begin{lstlisting}
f1(in) {
    in $\cup$ {(x:2)}
}
f2(in) {
    in $\cup$ {(y:3)}
}
f1 $\circ$ f2(in) {
    in $\cup$ {(x:2), (y:3)}
}
    \end{lstlisting}
\end{minipage}

\begin{minipage}{0.4\textwidth}
    \tt
    \captionsetup[]{type=figure}
    \begin{tikzpicture}[->,
        node distance = 12mm,
        start chain = going below,
        box/.style = {draw,rounded corners,fill=white,
            on chain,align=center, inner xsep=4 mm},
        unbox/.style = {on chain,align=center},
        edge/.style = {midway}]
        \node[on chain, inner xsep=5mm] (begin) {~};
        \node[box, below left=of begin](b2) {x = 2 \\ y = 3};
        \node[box, below right=of begin](b3) {y = 2 \\ x = 3};
        \node[on chain, below=of begin, below left=of b3, inner xsep=5mm] (end) {~};
        \path[]
            (begin) edge (b2)
            (begin) edge (b3)
            (b2) edge (end)
            (b3) edge (end);
    \end{tikzpicture}
    \captionof{figure}{Constant Propagation: Meet}
    \label{fig:m121:const_prop_meet}
\end{minipage}
\hfill
\begin{minipage}{0.5\textwidth}
  \begin{lstlisting}
f1(in) {
    in $\cup$ {(x,2), (y,3)}
}
f2(in) {
    in $\cup$ {(y,2), (x,3)}
}
f1 $\wedge$ f2 (in) {
    f1(in) $\wedge$ f2(in)
}
  \end{lstlisting}  
\end{minipage}

\subsubsection{Example: Available Expressions}

\begin{minipage}{0.4\textwidth}
    \tt
    \captionsetup{type=figure}
    \begin{tikzpicture}[->,
        node distance = 12mm,
        start chain = going below,
        box/.style = {draw,rounded corners,fill=white,
            on chain,align=center,inner xsep=8mm},
        unbox/.style = {on chain,align=center},
        edge/.style = {midway}]
        \coordinate[on chain] (begin);
        \node[box](b1) {f};
        \coordinate[on chain] (end) {};
        \path[]
            (begin) edge (b1)
            (b1) edge (end);
        \draw (b1.230) -| coordinate (l1) ++ (0.6,-0.3) -| coordinate (l2) ([xshift=5mm]b1.east) |-
        coordinate (l3) ([yshift=5mm]b1.50) -- coordinate (l4) (b1.50);
    \end{tikzpicture}
    \captionof{figure}{Available Expressions}

    \begin{lstlisting}
while x != n:
    x = x + 1
    \end{lstlisting}
\end{minipage}
\hfill
\begin{minipage}{0.5\textwidth}
    \begin{lstlisting}
f(in) {
    if in[x] != n
        in{x $\leftarrow$ in[x] + 1} 
            // Updation on in
    else
        in
}
    \end{lstlisting}

    Fallback Closure:
    \begin{lstlisting}
f* (in) {
    in {x $\leftarrow \bot$}
}
    \end{lstlisting}

    Symbolic Closure:
    \begin{lstlisting}
f* (in) {
    in {x $\leftarrow$ max(in(x), n)}
}
    \end{lstlisting}
\end{minipage}

\subsection{Composing GEN/KILL Transfers}

Consider the example of reaching definitions.
\begin{lstlisting}
f1(in) = gen1 $\cup$ (in - kill1)
f2(in) = gen2 $\cup$ (in - kill2)

f1 $\circ$ f2(in) = gen2 $\cup$ ((gen1 $\cup$ (in - kill1)) - kill2)
f1 $\wedge$ f2(in) = gen1 $\cup$ (in - kill1) $\cup$ gen2 $\cup$ (in - kill2)
f1* (in) = gen1 $\cup$ (in - kill2)
\end{lstlisting}

For closure, notice that \texttt{f1(f1(x)) = f1(x)}
