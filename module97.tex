\section {Semilattice Representations, Size, Product and Height}
\setlength{\parindent}{0pt}
(Prepared by Shivam Goyal)

\vspace{0.3cm}

\subsection{Semilattice Representation}
\begin{itemize}
    \item \textbf{Set of values $V$ and a meet operator \^{}}
    \item \textbf{Set of values $V$ and a partial order $\leq$}
    \item \textbf{A semilattice diagram} : which is a directed acyclic graph with nodes and edges. It can have potentially multiple Top and Bottom nodes, but they may not necessarily exist for eg in a semilattice with $V$ as set of integers and $max$ as the meet operator
\end{itemize}

\subsection{Semilattice Size}
For some DFA problems, size of the semilattice i.e. the number of elements in the set of values $V$ can become very large. For eg. it could be exponential in number of definitions for a program, if we consider the set of definitions as a value for the semilattice. As the set of values $V$ would be the power set of the set of definitions in the program for some DFA like available expressions.
\subsection{Product of Semilattices}
\begin{itemize}
    \item One idea considering large semilattices is to represent a larger semilattice using product of smaller semilattices.
    \item Consider 2 semilattices with sets of values as $V1$ and $V2$ then we can define a new semilattice as their product.This would contain values in the set $V1$ x $V2$. 
    \item The partial order for the product semilattice would be obtained by applying the respective partial order relations on both the elements of the tuple, i.e. for $\leq$ (the new partial order):
    \par
    ($x_{1}$,$x_{2}$) $\leq$ ($y_{1}$,$y_{2}$) iff $x_{1}$ $\leq_{1}$ $y_{1}$ and $x_{2}$ $\leq_{2}$ $y_{2}$ where $\leq_{1}$ and $\leq_{2}$ are the respective partial orders of the smaller semilattices. 
    \item Similar thing can be done in case of the meet operator with the respective meet operators being applied on each element of the tuple, i.e. for \^{} (the new meet operator):
    \par
    ($x_{1}$,$x_{2}$) \^{} ($y_{1}$,$y_{2}$) $=$ ($x_{1}$ \^{}$_{1}$ $y_{1}$ , $x_{2}$ \^{}$_{2}$ $y_{2}$), where \^{}$_{1}$ and \^{}$_{2}$ are the respective meet operators of the smaller semilattices.
\end{itemize}
