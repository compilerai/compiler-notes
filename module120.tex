\section{Computational Structure of a Region-Based DFA}
\FloatBarrier
\begin{flushright}
    \textit{Scribed by Arpit Saxena}
\end{flushright}

\begin{enumerate}
    \item {
        Identify transfer function of each region (and sub-region)

        \textbf{Bottom-up pass:}
        \begin{enumerate}
            \item Basic blocks transfer function derived from individual 
            statement's transfer functions 
            \item Loop body's transfer function from basic block's transfer 
            function
            \item Loop's from loop body's, \ldots
        \end{enumerate}
    }
    \item {
        Compute the dataflow values at each program point by using the summary
        transfer functions (for entire regions)

        \textbf{Top-down analysis}: Descend down the region hierarchy
        for internal points
    }

    
\end{enumerate}

\begin{figure}[H]
    \centering
    \subfloat[\label{fig:m120:ex1:a}] {{
        \tt
        \begin{tikzpicture}[->,
            node distance = 12mm,
            start chain = going below,
            box/.style = {draw,rounded corners,fill=white,
                on chain,align=center},
            unbox/.style = {on chain,align=center},
            edge/.style = {midway}]
            \coordinate[on chain] (begin);
            \node[box] (b1) {
                sum = 0\\
                Q1 --- \\
                i = 0 \\
                P1 ---
            };
            \node[box] (b2) {
                P2 --- \\
                i < n
            };
            \node[unbox,below left=of b2] (b3) {exit};
            \node[box,below=of b2] (b4) {sum += i \\ i++};
            \path[]
                (begin) edge (b1)
                (b1) edge (b2)
                (b2) edge node[edge, left] {F} (b3)
                    edge node[edge, right] {T} (b4);
            \draw (b4.230) -| ++(0.6,-0.3) -| ([xshift=5mm]b2.east) |-
            ([yshift=3mm]b2.50) -- (b2.50);
        \end{tikzpicture}
    }}
    \subfloat[\label{fig:m120:ex1:b}] {{
        \tt
        \begin{tikzpicture}[->,
            node distance = 12mm,
            start chain = going below,
            box/.style = {draw,rounded corners,fill=white,
                on chain,align=center},
            unbox/.style = {on chain,align=center},
            edge/.style = {midway}]
            \coordinate[on chain] (begin);
            \node[box](b1) {sum = 0 \\ i = 0};
            \node[box](b2) {loop body \\ region};
            \node[unbox](b3) {exit};
            \path[]
                (begin) edge (b1)
                (b1) edge (b2)
                (b2) edge (b3);
            \draw (b2.230) -| ++(0.6,-0.3) -| ([xshift=5mm]b1.east) |-
            ([yshift=5mm]b1.50) -- (b1.50);
            
        \end{tikzpicture}
    }}
    \subfloat[\label{fig:m120:ex1:c}] {{
        \tt
        \begin{tikzpicture}[->,
            node distance = 12mm,
            start chain = going below,
            box/.style = {draw,rounded corners,fill=white,
                on chain,align=center},
            unbox/.style = {on chain,align=center},
            edge/.style = {midway}]
            \coordinate[on chain] (begin);
            \node[box](b1) {sum = 0 \\ i = 0};
            \node[box](b2) {loop region};
            \node[unbox](b3) {exit};
            \path[]
                (begin) edge (b1)
                (b1) edge (b2)
                (b2) edge (b3);
        \end{tikzpicture}
    }}
    \subfloat[\label{fig:m120:ex1:d}] {{
        \tt
        \begin{tikzpicture}[->,
            node distance = 12mm,
            start chain = going below,
            box/.style = {draw,rounded corners,fill=white,
                on chain,align=center},
            unbox/.style = {on chain,align=center},
            edge/.style = {midway}]
            \coordinate[on chain] (begin);
            \node[box](b1) {Procedure \\ Region};
            \node[unbox](exit) {exit};
            \path[]
                (begin) edge (b1)
                (b1) edge (exit);
        \end{tikzpicture}
    }}
    \caption{A Region Hierarchy\\ Note that P1, P2, \ldots\ are program points}
    \label{fig:m120:ex1}
\end{figure}

In figure \autoref{fig:m120:ex1:b}, the loop body's transfer function
would be:

{
    \centering
    \tt
    f\textsubscript{b}(m) = m' \\
    m'(i) = m(i) + 1 \\
    m'(sum) = m(sum) + m(i) \\
}

Based on f\textsubscript{b}, we'll try to create a summary transfer
function for the entire loop.

In figure \autoref{fig:m120:ex1:c}, the loop's transfer function would be

{
    \tt
    \centering
    m\textsuperscript{l} = f\textsubscript{l}(m) \\
    m\textsuperscript{l}(i) = max(m(i), n) \\
    m\textsuperscript{l}(sum) = m(sum) + $\sum_{j=m(i)}^{n-1} j$ \\
}

Note that if initially $i \geq n$, then the loop is not even entered, which
is the reason for having max in the second equation.

All iterations of the loop have been summarised as a single transfer function
f\textsubscript{l}, and it is a sophisticated symbolic analysis which may not
be possible for every loop. In that case, we have a fallback option of just
executing a basic DFA.

For figure \autoref{fig:m120:ex1:d}, we use the fact that we know at start of loop
region that \texttt{m(i) = 0, m(sum) = 0} and get the following transfer
function

{
    \tt
    \centering
    m\textsuperscript{p} = f\textsubscript{p}(m) \\
    m\textsuperscript{p}(i) = n \\
    m\textsuperscript{p}(sum) = n * (n - 1) / 2 \\
}

Note that the transfer function doesn't need to be a closed form function. It
only needs to be a computable (terminating) function.
E.g.: Trivial solutions based on fixed point (terminating because the lattices
have finite height).

Suppose we wanted to compute value at P1 (refer to \autoref{fig:m120:ex1}).

In top-down fashion, we start from the procedure region. If we needed Q1,
then f\textsubscript{1} (which transfer function for the first basic block
in figure \autoref{fig:m120:ex1:c}) would only give summary of the basic block.
So, we'll apply transfer function of \texttt{sum = 0}.

In general, at every level of the hierarchy: Compute the DFA values on that level
of the hierarchy and we'll get boundary values of next-level regions. If we're
interested in an internal point of a region, we can use DFA value computed at 
beginning of the region to get value internal to the region.

To compute value at P\textsubscript{2}:
\begin{itemize}
    \item Run DFA on procedure region
    \item Get value at beginning of loop region (P1). Note it's different
    from P2 because the former is executed only once while the latter is entered
    at every loop iteration.
    \item Using P1, we find value at P1 at next level of the hierarchy.
    \[Val[P_1] \wedge f_b(f_1(m_0)) \wedge f_b^2(f_1(m_0)) \wedge \ldots\]
\end{itemize}